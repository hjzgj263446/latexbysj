	
\begin{chineseabstract}
现实生活中不断产生大量的短文本数据,例如新闻文本,美团、大众点评等网站上用户发表的针对某些方面的评论。文本数据的产生必然伴随着对数据的归类,如何提升分类
效率,减少人工成本,这便是文本数据分类的研究方向。此外从这些海量数据中挖掘出用户的情感,有助于精准地刻画用户,从而辅助平台进行针对性的提供服务。
然而,目前大多数方法都忽略了文本单词之间的联系或是方面词与上下文之间的联系,导致分类性能表现不好。

本文主要研究了基于图模型的文本分类算法,包括整体文本分类和方面级情感分析任务。通过图模型结合注意力机制挖掘单词之间、方面词与文本上下文之间的联系,学得更好的单词文本向量表示,从而提升模型性能。
本文主要的研究内容如下:

\begin{itemize}
    \item [1)] 
    提出了一种整体文本分类的算法。该方法将每个文本以单词为节点,单词之间的关系为边构建一个文本图。同时建立一个连接到所有单词节点的超节点,用以表示文本整体信息。随后采用一个带有注意力机制的图卷积神经网络学习超节点以及单词节点的向量表示,最后融合两者向量的信息,
    提升文本分类准确率。
    \item [2)] 
    提出了一种方面级情感分析的算法。首先将文本转化为图,然后构建一个连接方面词中所有单词的超节点。通过带有注意力机制和门控机制的图卷积神经网络学习单词向量表示和超节点向量表示。最后通过超节点向量作为方面词向量联合文本单词向量实现分类任务。同时还结合BERT预训练模型进一步
    提升分类性能。
\end{itemize}

总体来说,本文所提出的方法都基于图模型,通过图模型挖掘文本单词之间的联系,学习更好的单词向量表示。同时利用注意力机制和门控机制控制图模型中节点信息的传递过程。在大量数据集的验证下,
本文所提出的方法展现了较好的性能,能够用于处理短文本数据。

\chinesekeyword{文本分类,图模型,向量表示,方面词,注意力机制}
\end{chineseabstract}

